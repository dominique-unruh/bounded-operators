\title{A library about complex bounded operators in Isabelle/HOL}
\author{ 
        Dominique Unruh and Jos\'e M. Rodr\'iguez Caballero \\
                Institute of Computer Science \\ University of Tartu \\ Tartu, Estonia
}
\date{\today}

\documentclass[12pt]{article}

\usepackage[english]{babel}

\usepackage{blindtext}

\usepackage{amsmath}
\usepackage{hyperref}
\usepackage{lipsum}
\usepackage{amsfonts}
\usepackage{enumitem}
\usepackage{epigraph}
\usepackage{graphicx}
\usepackage{amsthm}

\newtheorem{thm}{Theorem}
\newtheorem{prop}{Proposition}
\newtheorem{cor}{Corollary}
\newtheorem{lem}{Lemma}
\newtheorem{claim}{Claim}


\theoremstyle{definition}
\newtheorem{defn}{Definition}


\begin{document}
\maketitle

\begin{abstract}
We develop a new library in Isabelle/HOL for the theory of complex bounded operators. Our library is an extension of earlier formalizations of the basics notions of functional analysis that were limited to the real numbers as the ground field. Our library contains a type classes corresponding to complex vector spaces, complex normed spaces, the complex Banach spaces, the complex Hilbert spaces and finite-dimensional versions. We define a type of $\ell^2$ sequences of complex numbers indexed by a given type. We introduce a type for bounded operators between two complex vector spaces. Then we prove the essential properties of these objects and we add code generation for the cases when computation is possible.
$\newline$
\begin{flushleft} 
\textbf{Categories and Subject Descriptors.} F.4.1 [Mathematical Logic]:
Mechanical theorem proving; G.4.2 [Certification and testing]; G.4.9 [Verification]
$\newline$

\textbf{Keywords.} complex Hilbert spaces, complex bounded operators, Isabelle/HOL, formal verification
\end{flushleft}
\end{abstract}


\section{Introduction}
In Unruh's quantum relational Hoare logic (qRHL) \cite{unruh2019quantum}, the complex Hilbert spaces and complex bounded operators between Hilbert spaces are the basic building blocks. These structures also appears in quantum information theory \cite{wilde2013quantum}, quantum mechanics \cite{prugovecki1982quantum}, machine learning \cite{zhang2012refinement}, partial differential equations \cite{picard2011partial} and several areas of mathematical analysis \cite{conway2019course}, e.g., Fourier analysis and functional analysis. Previously to the present work, in the default libraries of Isabelle/HOL and in Archives of Formal Proofs there was a development of functional analysis mainly in the direction of real analysis \cite{fleuriot2000mechanization, fleuriot2000mechanizing, holzl2011three, holzl2013type, boldo2016formalization}
 and of linear algebra mainly in the direction of finite-dimensional vector spaces over an arbitrary ground field \cite{thiemann2016formalizing}. Also, there were some results in complex analysis \cite{li2020evaluating, eberl2019nine, li2019counting} formalized in Isabelle/HOL, but (infinite-dimensional) complex functional analysis was absent, despite its importance in quantum theory.

Our new library for complex functional analysis in Isabelle/HOL derives from the
earlier libraries on real mathematical analysis (including nonstandard analysis), real functional analysis and finite dimensional linear algebra, but brings them closer to the goal of formal verification of quantum cryptography: our ultimate goal is to embed Unruh's qRHL into Isabelle/HOL, using Hilbert spaces, bounded operators and tensor products as basic building blocks. The missing ingredient, the tensor products, are work-in-progress\footnote{\url{https://github.com/dominique-unruh/tensor-product}} and they will be introduced in a future library.

In the next lines we describe our main contribution to the development of analysis in Isabelle/HOL. We will omit some of our results in order to avoid an overload of information in the description of our work.

\begin{enumerate}[label*=\Alph*.]
\item The main type classes that we introduced are the following ones, ordered in decreasing order of generality (each class is a subclass of the classes above, classes at the same level are incomparable).
\begin{enumerate}[label*=\arabic*.]
\item \textsf{complex\_vector} (vector spaces over the complex numbers),
\item \textsf{complex\_normed\_vector} (normed space over the complex numbers), 
\item \textsf{cbanach\_space} (Banach space over the complex number) and \textsf{complex\_inner\_product} (vector space endowed with a sesquilinear form),
\item \textsf{chilbert\_space} (Hilbert space over the complex numbers),
\item \textsf{onb\_enum} (finite dimensional Hilbert space over the complex numbers with an identified orthonormal basis).
\end{enumerate}
\item We introduced the (parametric) type of \emph{square-summable} families of complex numbers indexed by a type $\textsf{'a}$, denoted $\textsf{'a ell2}$, which is an instantiation of the type class $\textsf{chilbert\_space}$.
\item We introduced the (parametric) type of \emph{complex bounded operator} between two complex normed spaces $\textsf{'a}$ and  $\textsf{'b}$, denoted $\textsf{'a} \Rightarrow_{\textsf{CL}} \textsf{'b}$, which is, under some conditions, is an instantiation of the type class \textsf{cbanach\_space}.
\item We developed code generation for the transformation of a bounded operator of type $\textsf{'a}::\textsf{onb\_enum} \Rightarrow_{\textsf{CL}} \textsf{'b}::\textsf{onb\_enum}$ into a matrix of type $\textsf{complex mat}$ (this type was defined in \cite{Jordan_Normal_Form-AFP}).
\item We extended the existing structures of nonstandard analysis in order to include the complex numbers and we used techniques from nonstandard analysis some of our proofs.
\end{enumerate}

The aim of the present paper is to introduce our library of complex bounded operators to an audience satisfying the following preliminaries. On the side of mathematics, we assume basic knowledge of the basic functional analysis \cite{conway2019course}, essentially the concepts of Hilbert space and bounded operators. On the side of programming, we assume basic knowledge of the proof assistant Isabelle/HOL \cite{nipkow2002isabelle}, in particular, the Isar language for formal proofs.

\section{Complex vector spaces}
\blindtext[6]

\section{Complex inner product spaces}
\blindtext[6]

\section{Complex $\ell^2$-spaces}
\blindtext[6]

\section{Bounded operators}
\blindtext[6]

\section{Code generation}
\blindtext[6]


\section{Summary}
We applied the type classes and the type abstraction (typedef) from Isabelle/HOL in order to formalize the types appearing in the theory of complex bounded operators. We reproduced the complex analog of the hierarchy of type classes from the already existing resources about real bounded operators and real vector spaces. Our intention is to decrease the learning time for an user who already has skills in the existing libraries.

The main difference between the formalization of the theory of complex bounded operators and the already existing real case is that in the type class \textsf{real\_inner\_product} there is a bilinear form representing the complex inner product whereas in \textsf{complex\_inner\_product}  the bilinear form is substituted by a sesquilinear form. So, even if the complex inner product is a real inner product when restricted to a real ground field, this property cannot be expressed in Isabelle/HOL in the framework of our formalization.

Even considering this last limitation, the present work proves that the basic theory of functional analysis can be formalized in Isabelle/HOL in a rather natural way.

\section{Future work}
Our next project is to develop a library for the Hilbert tensor product between two Hilbert spaces. This library will involve the bounded operators that are induced between tensor products by the bounded operators between the respective (tensor) factors. Once all the mathematical preliminaries will be ready, we will embed Unruh's qRHL into Isabelle/HOL and explore its potential in order to do formal verification of protocols from quantum cryptography.


\section*{Acknowledgments}
The second author is supported by the Air Force Office of Scientific Research
through the project ``Verification of quantum cryptography" (AOARD Grant
FA2386-17-1- 4022).


\bibliographystyle{alpha}
\bibliography{mybibfile}

\end{document}


