\documentclass[11pt,a4paper]{article}
\usepackage{isabelle,isabellesym}


%this should be the last package used
\usepackage{pdfsetup}

% urls in roman style, theory text in math-similar italics
\urlstyle{rm}
\isabellestyle{it}

\begin{document}

\title{The $p$-adic norm and its application to the harmonic numbers}
\author{Jos\'e Manuel Rodr\'iguez Caballero}
\maketitle

\begin{abstract}
Let $H_n = 1 + \frac{1}{2} + \frac{1}{3} + \frac{1}{4} + \frac{1}{5} + ... + \frac{1}{n}$ be the $n$-th harmonic number.
L. Taeisinger \cite{theisinger1915bemerkung} proved in 1915 that $H_n$ is not an integer for $n \geq 2$. K\"ursch\'ak \cite{kurschak1918harmonic} proved in 1918 that $H_n - H_m$ is not an integer provided that $0 \leq m \leq n-2$. In this paper, we will introduce the $p$-adic norm and we will use it in order to derive these results.
\end{abstract}

\tableofcontents

\input{session}


\bibliographystyle{abbrv}
\bibliography{root}

\end{document}
